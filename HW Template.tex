%=======================02-713 LaTeX template, following the 15-210 template==================
%
% You don't need to use LaTeX or this template, but you must turn your homework in as
% a typeset PDF somehow.
%
% How to use:
%    1. Update your information in section "A" below
%    2. Write your answers in section "B" below. Precede answers for all 
%       parts of a question with the command "\question{n}{desc}" where n is
%       the question number and "desc" is a short, one-line description of 
%       the problem. There is no need to restate the problem.
%    3. If a question has multiple parts, precede the answer to part x with the
%       command "\part{x}".
%    4. If a problem asks you to design an algorithm, use the commands
%       \algorithm, \correctness, \runtime to precede your discussion of the 
%       description of the algorithm, its correctness, and its running time, respectively.
%    5. You can include graphics by using the command \includegraphics{FILENAME}
%
\documentclass[11pt]{article}
\usepackage{amsmath,amssymb,amsthm}
\usepackage[noend]{algpseudocode}
\usepackage{graphicx}
\usepackage[margin=1in]{geometry}
\usepackage{mathtools}
\usepackage{fancyhdr}
\usepackage{algorithmicx}
\setlength{\parindent}{0pt}
\setlength{\parskip}{5pt plus 1pt}
\setlength{\headheight}{13.6pt}
\newcommand\question[2]{\vspace{.25in}\hrule\textbf{#1: #2}\vspace{.5em}\hrule\vspace{.10in}}
\renewcommand\part[1]{\vspace{.10in}\textbf{(#1)}}
\newcommand\analysis{\vspace{.10in}\textbf{Analysis: }\newline}
\newcommand\outline{\vspace{.10in}\textbf{Proof Outline: }\newline}
\newcommand\algorithm{\vspace{.10in}\textbf{Algorithm: }}
\newcommand\correctness{\vspace{.10in}\textbf{Correctness: }\newline}
\newcommand\runtime{\vspace{.10in}\textbf{Running time: }\newline}
\newcommand\definitions{\emph{Definitions:}\newline}
\newcommand\claim{\emph{Claim.}\newline}
\pagestyle{fancyplain}
\lhead{\textbf{\NAME}}
\chead{\textbf{CS 289A Midterm Problem Set}}
\rhead{\today}
\begin{document}\raggedright
%Section A==============Change the values below to match your information==================
\newcommand\NAME{Siddharth Joshi}  % your name
\newcommand\HWNUM{}              % the homework number
%Section B==============Put your answers to the questions below here=======================

% no need to restate the problem --- the graders know which problem is which,
% but replacing "The First Problem" with a short phrase will help you remember
% which problem this is when you read over your homeworks to study.

\question{1}{Deterministic Communication Complexity of Graph Isomorphism} 

The problem of Graph Isomorphism naturally seems very similar to the Equality function: a function we are very familiar with. Moreover, due to the inherent difficulty in accurately deciphering the pattern in the characteristic matrix for Graph Isomorphism, the approach taken was to reduce Equality of $n^2$ bits to the the problem of Isomorphism of Simple Undirected Graphs on n vertices. \newline

\definitions
Let $\mathbf{G_n}$ = the set of simple undirected graphs on n vertics \newline
Let $IS_n: \mathbf{G_n} \times \mathbf{G_n} \to \{0, 1\}$ be the graph isomorphism function i.e. 1 if the Graph x is an isomorphism of Graph y and 0 otherwise
\newline

\claim
$ D(IS_n) \geq n^2 $

\proof 
Reduction from equality of $n^2$ bit strings to isomorphism of simple undirected graphs on n vertices. (Note in the reduction can only use local work i.e. local work is free as Alice and Bob are computationally unbounded entities) \newline
Show that the number of unique simple undirected graphs under isomorphism on n vertices are $\theta2^{(n^2)}$ and therefore each can be represented using $\theta(n^2)$ bits. Therefore the $n^2$ bits of equality can be interpreted as graphs $\in \mathbf{G_n}$ and isomorphism $\leftrightarrow$ equality.
\newpage

\question{2}{Deterministic Communication Complexity of Determining nth least significant bit of the product of two natural numbers} 

Product of n bits strings (do not need to consider entire number)

\definitions
Let $Product_n: {\{0, 1\}}^n \times {\{0, 1\}}^n \to \{0, 1\}$ be the the function on n bits strings indicating the nth least significant bit of the product of the integers represented by the strings

\claim
$D(Product_n) \geq n$ (maybe plus 1 look at the inequality)

\proof 
submatrix embedding 

other pattern

\newpage

\question{3}{Deterministic Communication Complexity of Computing a Boolean formula on n variables} 

some info

\definitions
Let $Boolean_n: {\{0, 1\}}^n \times {\{0, 1\}}^n \to \{0, 1\}$

\claim
$D(boolean_n) = \omega(n)$ 

\proof 

no idea as of now 

\newpage

\end{document}










