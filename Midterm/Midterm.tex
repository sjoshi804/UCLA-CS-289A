%=======================02-713 LaTeX template, following the 15-210 template==================
%
% You don't need to use LaTeX or this template, but you must turn your homework in as
% a typeset PDF somehow.
%
% How to use:
%    1. Update your information in section "A" below
%    2. Write your answers in section "B" below. Precede answers for all 
%       parts of a question with the command "\question{n}{desc}" where n is
%       the question number and "desc" is a short, one-line description of 
%       the problem. There is no need to restate the problem.
%    3. If a question has multiple parts, precede the answer to part x with the
%       command "\part{x}".
%    4. If a problem asks you to design an algorithm, use the commands
%       \algorithm, \correctness, \runtime to precede your discussion of the 
%       description of the algorithm, its correctness, and its running time, respectively.
%    5. You can include graphics by using the command \includegraphics{FILENAME}
%
\documentclass[11pt]{article}
\usepackage{amsmath,amssymb,amsthm}
\usepackage[noend]{algpseudocode}
\usepackage{graphicx}
\usepackage[margin=1in]{geometry}
\usepackage{mathtools}
\usepackage{fancyhdr}
\usepackage{algorithmicx}
\usepackage{listings}
\setlength{\parindent}{0pt}
\setlength{\parskip}{5pt plus 1pt}
\setlength{\headheight}{13.6pt}
\newcommand\question[2]{\vspace{.25in}\hrule\textbf{#1: #2}\vspace{.5em}\hrule\vspace{.10in}}
\newcommand\newAppendix[2]{\vspace{.25in}\hrule\textbf{#1 #2}\vspace{.5em}\hrule\vspace{.10in}}
\renewcommand\part[1]{\vspace{.10in}\textbf{(#1)}}
\newcommand\analysis{\vspace{.10in}\emph{Analysis: }\newline}
\newcommand\outline{\vspace{.10in}\textbf{Proof Outline: }\newline}
\newcommand\algorithm{\vspace{.10in}\emph{Algorithm: }}
\newcommand\correctness{\vspace{.10in}\emph{Correctness: }\newline}
\newcommand\runtime{\vspace{.10in}\emph{Running time: }\newline}
\newcommand\problem{\emph{Problem Statement:}\newline}
\newcommand\definitions{\emph{Definitions:}\newline}
\newcommand\claim{\emph{Claim.}\newline}
\pagestyle{fancyplain}
\lhead{\textbf{\NAME}}
\chead{\textbf{CS 289A Midterm Problem Set}}
\rhead{\today}
\begin{document}\raggedright
%Section A==============Change the values below to match your information==================
\newcommand\NAME{Siddharth Joshi}  % your name
\newcommand\HWNUM{}              % the homework number
%Section B==============Put your answers to the questions below here=======================

% no need to restate the problem --- the graders know which problem is which,
% but replacing "The First Problem" with a short phrase will help you remember
% which problem this is when you read over your homeworks to study.

\question{1}{Isomorphic Graphs} 

\problem
Alice and Bob each have an undirected graph on n
vertices. How much communication do they need to find out deterministically if
their graphs are isomorphic?

\analysis
The problem of Graph Isomorphism naturally seems very similar to the Equality function: a function we are very familiar with. Moreover, due to the inherent difficulty in accurately deciphering the pattern in the characteristic matrix for Graph Isomorphism, the approach taken was to reduce Equality of $n^2$ bits to the the problem of Isomorphism of Simple Undirected Graphs on n vertices. \newline

\definitions
Let $\mathbf{G_n}$ = the set of simple undirected graphs on n vertics \newline
Let $IS_n: \mathbf{G_n} \times \mathbf{G_n} \to \{0, 1\}$ be the graph isomorphism function i.e. 1 if the Graph x is an isomorphism of Graph y and 0 otherwise
\newline

\claim
$ D(IS_n) \geq n^2 $

\proof 
Reduction from equality of $n^2$ bit strings to isomorphism of simple undirected graphs on n vertices. (Note in the reduction can only use local work i.e. local work is free as Alice and Bob are computationally unbounded entities) \newline
Show that the number of unique simple undirected graphs under isomorphism on n vertices are $\theta2^{(n^2)}$ and therefore each can be represented using $\theta(n^2)$ bits. Therefore the $n^2$ bits of equality can be interpreted as graphs $\in \mathbf{G_n}$ and isomorphism $\leftrightarrow$ equality.
\newpage

\question{2}{Integer Multiplication} 
\problem
How much deterministic communication does it take to compute the n
least significant bit of the product of two natural numbers, of which Alice has one and Bob the other?

\analysis
Product of n bits strings (do not need to consider entire number)

\definitions
Let $Product_n: {\{0, 1\}}^n \times {\{0, 1\}}^n \to \{0, 1\}$ be the the function on n bits strings indicating the nth least significant bit of the product of the integers represented by the strings

\claim
logn bound using powers of 2 being distinct s

\claim
$D(Product_n) \geq n + 1$ (maybe plus 1 look at the inequality)

\emph{The kernel has always dimension $< 2^{n - 1}$ vectors therefore by rank nullity}

proof for the above statement:

Argue that vectors of the desired form span the nullspace 

Show that there can exist only $< 2 ^ {n - 1}$ such linearly independent vectors and hence nullity $< 2 ^ {n - 1}$

Use rank nullity and hence rank $> 2 ^ {n - 1}$

Thus conclude. 

\emph{rank $>  2^{n - 1} \implies$ log rank $>$ n - 1 $\implies$ ceiling log rank $\geq$ n and hence D(f) $\geq$ 1}

\newpage

\question{3}{Boolean Formulas} 

\problem
A Boolean formula in variables z1, . . . , zn is a fully
parenthesized expression with operands $z1, \lnot z1, . . . , zn, \lnot zn$ and operators $\land$ and $\lor$. Let $\phi (z1, . . . , zn)$ be a Boolean formula in which every variable occurs exactly once. Prove that computing $\phi (x \oplus y$ deterministically on input $x, y \in {0, 1}^n$
requires
$\phi(n)$ bits of communication.

\analysis
Describe the boolean function and the requirements it imposes

\definitions
Let $Boolean_n: {\{0, 1\}}^n \times {\{0, 1\}}^n \to \{0, 1\}$

\claim
$D(Boolean_n) = n + 1$ 

\proof 
For each bit there are two strings that differ in value of just that bit and have different outputs thus must be separated by any protocol

Thus there exist n such distinct partitioning cuts (note that the partitions must necessarily be distinct as they are not this implies that the 2 strings corresponding to bit i differ in bit j as well - contradiction)

Any valid protocol must induce these n partitions (as for any given bit, it is impossible for Alice or Bob to know precisely which side of the partition the final answer lies in with just their input for all possible input strings)

To induce these n distinct partitions any protocol must send atleast n bits and hence the deterministic commmunication complexity of any such boolean formula is exactly n + 1 bits. 

\newpage

\question{4}{Jazzy Inner Product}

\problem
Define $f: {\{0, 1\}}^n \times {\{0, 1\}}^n \to \{0, 1\}$ by $f(x, y) = 1 \iff \sum{x_i \cdot y_i} = 0 \text{ mod (18181)}.$ What is the nondeterministic communication complexity of f?

\analysis
compare to known problem 

\definitions
Let $JIP_n: {\{0, 1\}}^n \times {\{0, 1\}}^n \to \{0, 1\}$

\claim
$N(JIP_n) = \Omega(n)$ 

\proof 
Maximal rectangles -> cross of the boolean vectors in subspace A of dimension n - k and subspace B of dimension k 

Number of boolean vectors in A = $2^{n - k}$ and number of boolean vectors in B = $2^{k}$ therefore total number of elements in any rectangle $\leq 2^{n - k} \cdot 2^{k} = 2^n$ 

Identical proof here on forth

\newpage

\question{5}{Relative Primality}

\problem
Alice and Bob’s inputs are integers a and b, respectively, where $a, b \in [1, 2^n]$. Prove that $\Theta(n/ log n)$ bits are necessary and sufficient to verify nondeterministically that a and b are relatively prime.

\analysis
... 

\definitions
Let $RP_n: {\{0, 1\}}^n \times {\{0, 1\}}^n \to \{0, 1\}$

\claim
$N(RP_n) = \Omega(n/log(n))$ 

\proof 
Use prime number theorem

Not really sure how to progress from there, going to write program to see matrices, but need to come up with protocol in order to prove upper bound

\newpage

\question{6}{Orthogonal Subspaces}

\problem
On input linear subspaces $A, B \subseteq F^n_2$, prove that $\Theta(n^2)$ bits of nondeterministic communication are necessary and sufficient to check if A and B are orthogonal.

\analysis
$N^2$ bits for the representation of the Subspace

\definitions
Let $OS_n: {\{0, 1\}}^{n^2} \times {\{0, 1\}}^{n^2} \to \{0, 1\}$

\claim
$N(OS_n) = \Omega(n^2)$ 

\proof 
Large fooling set

The fooling set contains all vector spaces and their orthogonal complement. 

By definition of orthogonal complement, it is the subspace containing all vectors orthogonal to a susbspace and hence any other all crosspoints will equal 0 (thus a valid fooling set)

Number of subspaces of ${\mathbb{F}_2}^n = 1 + {2^n \choose n - 1} + ... + {2^n \choose 1} + 1$

Largest term is by far second term as terms reduce in magnitude now as they $n << 2^{n-1}$

Therefore fooling set = $\Omega{2^n \choose n - 1} = \Omega({2^{n^2}})$

Therefore $N(OS_n) = \Omega(log(2^{n^2})) = \Omega(n^2)$

And by the trivial protocol of sending the entire input it is easy to see that this bound is tight
\newpage

\question{7}{Communication v/s Randomness}

\problem
Prove that any randomized protocol for EQn with probability of correctness 2/3 and communication cost c must use more than $log_2(n/c)$ bits of randomness.

\analysis
say something

\definitions
Let $EQ_n: {\{0, 1\}}^{n^2} \times {\{0, 1\}}^{n^2} \to \{0, 1\}$

Let P be a particular randomized protocol for $EQ_n$ with cost c with probability of correctness = 2/3

Let $B_P$ = number of random bits used by protocol P 

\claim
$B_{P} \geq log(n/c)$

\proof 

\newpage
\question{8}{Better Than Random}

\problem
Prove that every $f : {\{0, 1\}}^n \times {\{0, 1\}}^n \to \{0, 1\}$ has a randomized protocol with constant cost and error at most $1/2 - \Theta(2^{-n/2})$

\analysis
Thoughts

\definitions
Let $EQ_n: {\{0, 1\}}^{n^2} \times {\{0, 1\}}^{n^2} \to \{0, 1\}$

Let P be a particular randomized protocol for $EQ_n$ with cost c with probability of correctness = 2/3

Let $B_P$ = number of random bits used by protocol P 

\claim
$B_{P} \geq log(n/c)$

\proof 
\newpage
\newAppendix{Appendix:}{Python Script used for Empirical Data for Rank in Various Fields}
\begin{lstlisting}
from sage.all import *
    num_of_bits = int(raw_input("How many bits?\n"))
    m_f = []
    for x in range(0, 2**num_of_bits):
        l1 = []
        for y in range(0, 2**num_of_bits):
            z = x * y
            binary = ("{0:b}".format(z))[::-1]
            if (len(binary) < num_of_bits):
                l1.append(0)
            else:
                l1.append(int(binary[num_of_bits - 1]))
        m_f.append(l1)
    field_num = 2**(num_of_bits - 1)
    print(2*field_num)
    matrix_f = (matrix(GF(2), m_f))
    print(matrix_f.rank())
    matrix_f = (matrix(GF(2**num_of_bits), m_f))
    print(matrix_f.rank()) 
    matrix_f = (matrix(GF(2**(num_of_bits+1)), m_f))
    print(matrix_f.rank())
    matrix_f = (matrix(m_f))
    print(matrix_f.rank())
    kernel_f = (kernel(matrix_f)).matrix()
    print(kernel_f)
    hypothesis = True
    row_num = 0
    for row in kernel_f:
        pos = 0
        sum = 0
        abs_sum = 0
        digits = 0
        for element in row:
            sum += ((pos*element))
            abs_sum += abs((pos * element))
            pos += 1
            if element != 0:
                digits += 1
        if sum != 0:
            hypothesis = False
            break
        print(str(row_num) + ":" + str(digits) + " " + str(abs_sum))
        row_num += 1
    print(hypothesis) 
       
\end{lstlisting}
\end{document}










