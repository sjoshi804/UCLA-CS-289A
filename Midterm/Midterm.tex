%=======================02-713 LaTeX template, following the 15-210 template==================
%
% You don't need to use LaTeX or this template, but you must turn your homework in as
% a typeset PDF somehow.
%
% How to use:
%    1. Update your information in section "A" below
%    2. Write your answers in section "B" below. Precede answers for all 
%       parts of a question with the command "\question{n}{desc}" where n is
%       the question number and "desc" is a short, one-line description of 
%       the problem. There is no need to restate the problem.
%    3. If a question has multiple parts, precede the answer to part x with the
%       command "\part{x}".
%    4. If a problem asks you to design an algorithm, use the commands
%       \algorithm, \correctness, \runtime to precede your discussion of the 
%       description of the algorithm, its correctness, and its running time, respectively.
%    5. You can include graphics by using the command \includegraphics{FILENAME}
%
\documentclass[11pt]{article}
\usepackage{amsmath,amssymb,amsthm}
\usepackage[noend]{algpseudocode}
\usepackage{graphicx}
\usepackage[margin=1in]{geometry}
\usepackage{mathtools}
\usepackage{fancyhdr}
\usepackage{algorithmicx}
\usepackage{listings}
\setlength{\parindent}{0pt}
\setlength{\parskip}{5pt plus 1pt}
\setlength{\headheight}{13.6pt}
\newcommand\question[2]{\vspace{.25in}\hrule\textbf{#1: #2}\vspace{.5em}\hrule\vspace{.10in}}
\newcommand\newAppendix[2]{\vspace{.25in}\hrule\textbf{#1 #2}\vspace{.5em}\hrule\vspace{.10in}}
\renewcommand\part[1]{\vspace{.10in}\textbf{(#1)}}
\newcommand\analysis{\vspace{.10in}\emph{Analysis: }\newline}
\newcommand\outline{\vspace{.10in}\textbf{Proof Outline: }\newline}
\newcommand\algorithm{\vspace{.10in}\emph{Algorithm: }}
\newcommand\correctness{\vspace{.10in}\emph{Correctness: }\newline}
\newcommand\runtime{\vspace{.10in}\emph{Running time: }\newline}
\newcommand\problem{\emph{Problem Statement:}\newline}
\newcommand\definitions{\emph{Definitions:}\newline}
\newcommand\claim{\emph{Claim.}\newline}
\pagestyle{fancyplain}
\lhead{\textbf{\NAME}}
\chead{\textbf{CS 289A Midterm Problem Set}}
\rhead{\today}
\begin{document}\raggedright
%Section A==============Change the values below to match your information==================
\newcommand\NAME{Siddharth Joshi}  % your name
\newcommand\HWNUM{}              % the homework number
%Section B==============Put your answers to the questions below here=======================

% no need to restate the problem --- the graders know which problem is which,
% but replacing "The First Problem" with a short phrase will help you remember
% which problem this is when you read over your homeworks to study.

\question{1}{Isomorphic Graphs} 

\problem
Alice and Bob each have an undirected graph on n
vertices. How much communication do they need to find out deterministically if
their graphs are isomorphic?

\analysis
The problem of Graph Isomorphism by virtue of being an equivalence relation is very similar to that of equality. In particular, it is easy to construct a fooling set of size equal to the number of equivalence classes. Hence it remains to show that there exist many equivalence classes for the existence of a large fooling set. \newline

\definitions
Let $\mathbf{G_n}$ = the set of unique (up to isomorphisms) simple undirected graphs on n vertics \newline
Let $IS_n: \mathbf{G_n} \times \mathbf{G_n} \to \{0, 1\}$ be the graph isomorphism function i.e. 1 if the Graph x is an isomorphism of Graph y and 0 otherwise
\newline

\claim
$ D(IS_n) \geq \frac{n^2}{2} - \Omega(n(logn))) $

\proof 
Number of unique simple undirected graphs (allowing for double counting of graphs that are isomorphic to each other) on n vertices = $2^{n \choose 2}$

But every graph is isomorphic to at most $n!$ other graphs

Thus $\vert\mathbf{G_n}\vert \geq \frac{1}{n!} \cdot 2^{{n \choose 2}}$

Since the function is an equivalence relation $f_s(IS_n) = \vert\mathbf{G_n}\vert \geq \frac{1}{n!} \cdot 2^{{n \choose 2}}$

Thus $D(IS_n) \geq log( \frac{1}{n!} \cdot 2^{{n \choose 2}}) = {n \choose 2} - log(n!) = \frac{n^2}{2} - \Theta({nlog(n)})$

This lower bound is tight as there is a protocol with this cost. In particular, if Alice and Bob (being computationally unbounded entities) simply match the graph they receive to a graph $\in \mathbf{G_n}$ then running equality on the $(\frac{n^2}{2} - \Theta({nlog(n)}))$ bit representation of the graphs will be sufficient for computing $IS_n$

\newpage

\question{2}{Integer Multiplication} 
\problem
How much deterministic communication does it take to compute the $\text{n}^{\text{th}}$
least significant bit of the product of two natural numbers, of which Alice has one and Bob the other?

\analysis
The first simplification that can be done on this problem is reducing the problem from multiplication of any two natural numbers to the mutliplication of the integers represented by 2 n-bit strings (that are in particular the least significant n-bits of the original natural numbers' representation) as any bit more significant cannot affect the output of this function. 

\definitions
Let $Product_n: {\{0, 1\}}^n \times {\{0, 1\}}^n \to \{0, 1\}$ be the the function on n bits strings indicating the nth least significant bit of the product of the integers represented by the strings

\claim
$D(Product_n) \geq log(n)$ 

\proof
This proof uses the rank bound technique and shows that the characteristic matrix of the function $M_{Product_n}$ has rank $\geq n$ to arrive at the lower bound for the commmunication complexity of the function. 

Assuming rows and columns are arranged in lexicographic order it easy to see that 
the rows corresponding to powers of $2 \in [0, 2^n]$ are linearly independent. 

An inductive argument proves this by showing that for any $k < 2^n$ the row corresponding to kth power of 2 must be linearly independent of the previous k - 1 rows corresponding to powers of 2 as the first $2^{n-k-1}$ numbers have 0s in their k - 1 most significant bits thus must have 0s the columns corresponding to them in all the previous k - 1 rows but necessarily half of them must have a 1 in the kth most significant bit and thus will have 1s in the row correspoding to $2^k$,thus this row is linearly independent of the previous k - 1 rows (corresponding to powers of 2). A similar argument works for the base case in showing that rows corresponding to $2^0$ and $2^1$ are linearly independent. 

Thus $rank(M_{Product_n}) \geq log(2^n) = n$ 

Therefore by the rank bound $D(f) \geq log(rank(M_{Product_n})) \geq log(n)$\newline 


\claim
$D(Product_n) = n + 1$ 

\proof
Proving first $dim(kernel(M_{Product_n})) < 2^{n - 1}$

Conjecture: Vectors of the following form completely span the null space:
1 in the $a^{th}$ and $b^{th}$  positions and -1 in the $c^{th}$ and $d^{th}$ positions where $a$ and $b$ are such that $a + b = 2^n$, $c = 2^{n-1} - 1$ and $d = 2^{n-1} + 1$ positions 

I was unable to prove this conjecture (perhaps because this is a consequence of a deeper phenomenon), however the intuition behind this relies on the following argument. Moreover, there was overwhelming evidence from empirical data that conformed to this hypothesis (the appendix contains the python script used in order to obtain this empirical data):

The vectors in the nullspace are a subset of the linear combinations of the vectors that represent in essence multiplication by 0. This is to say that the vectors of the form mentioned above are vectors where the 4 indices with non-zero values along with values sum up to 0 i.e. $a + b - c - d = 0$. The choice for 4 values of this form is not arbitrary as it can be argued easily that any other vector of such a form (sum to 0) in the nullspace can be expressed as a linear combination of these vectors, an the choice of 4 in particular allows for easy analysis of the possible vectors of this type. Since $a \in [1, 2^{n - 1} - 2]$ the number of vectors of this form are strictly less than $2^{n - 1}$ and thus $dim(kernel(M_{Product_n})) < 2^{n - 1}$

By the rank nullity theorem, this implies $rank(M_{Product_n}) > 2 ^ {n - 1}$

Thus by the rank bound, $D(Product_n) \geq \lceil log(rank(M_{Product_n})) \rceil + 1 \geq n + 1$

\newpage

\question{3}{Boolean Formulas} 

\problem
A Boolean formula in variables z1, . . . , zn is a fully
parenthesized expression with operands $z1, \lnot z1, . . . , zn, \lnot zn$ and operators $\land$ and $\lor$. Let $\phi (z1, . . . , zn)$ be a Boolean formula in which every variable occurs exactly once. Prove that computing $\phi (x \oplus y)$ deterministically on input $x, y \in {0, 1}^n$
requires
$\Omega(n)$ bits of communication.

\analysis
The requirement that every operand appear exactly once ensures dependence on every operand by eliminating all redundancy from the function and as such this observation can be used to justify a tight lower bound. 

\definitions
Let $Boolean_n: {\{0, 1\}}^n \times {\{0, 1\}}^n \to \{0, 1\}$ be any boolean formula of the form described in the problem statement. 

Let $z = x \oplus y$ and $z_i$ be the $i^{th}$ bit of z

\claim
$D(Boolean_n) = n + 1$ 

\proof 
$\exists$ Two possible values of z s.t. that they differ only in $z_i$ but do not have the same output $\forall i$

This follows from every operand appearing exactly once as in the $2^n$ possible values for z, there must exists two strings with n - 1 bits with values such that the value of the formula depends entirely on $z_i$ (as the n - 1 bits can take all possible values and since the formula cannnot contain any constants, the formula with values substituted in for the n - 1 bits cannot be constant and in particular can be reduced to either $z_i$ or $\lnot z_i$)

Hence any protocol must induce a partition in the characteristic matrix splitting elements corresponding to these values.

But since this is true $\forall i$ there exist n such distinct\footnote{Note that the partitions must necessarily be distinct as they are not this implies that the 2 strings corresponding to bit i differ in bit j as well - contradiction} partitioning cuts. 

Any valid protocol must induce these n partitions (as for any given bit, there exists x and y such that it is impossible for Alice or Bob to know precisely which side of the corresponding partition the final answer lies in solely on the basis of $x_i$ or $y_i$ respectively\footnote{The characteristic matrix for the xor function with full rank evidences this})

To induce these n distinct partitions any protocol must send at least n bits.

Moreover, since the boolean formula cannot contain constants and due to the aformentioned observations, every boolean formula of this form must also necessarily be non-constant. 

Hence $D(Boolean_n) = n + 1$ 

\newpage

\question{4}{Jazzy Inner Product}

\problem
Define $f: {\{0, 1\}}^n \times {\{0, 1\}}^n \to \{0, 1\}$ by $f(x, y) = 1 \iff \sum{x_i \cdot y_i} = 0 \text{ mod (18181)}.$ What is the nondeterministic communication complexity of f?

\analysis
This problem seems identical to the analysis for modulo 2 Inner Product function discussed in class and since the 18181 is a prime number, it allows for a similar analysis of rectangle size. 

\definitions
Let $JIP_n: {\{0, 1\}}^n \times {\{0, 1\}}^n \to \{0, 1\}$ be the function $f$ described in the problem statement

\claim
$N(JIP_n) = \Omega(n)$ 

\proof 
Note that in any 1-monochromatic rectangle $R = A \times B$, $A$ and $B$ can be interpreted as subspaces\footnote{in the finite field with 18181 elements} in $\mathbb{F}_{18181}$ and therefore can be enlarged to $R' = span(A) \times span(B)$

Let $A' = span(A)$ and $B' = span(B)$

Let $dim(A') = k$ and since $B'$ is orthogonal to $A'$ in $\mathbb{F}_{18181}$, it can have dimension at most that of the orthogonal complement of $A'$ and thus $dim(B') \leq n - k$

Therefore $\lvert A' \rvert =$ number of boolean vectors in $A' = 2^k$

Similarly, $\lvert B' \rvert =$ number of boolean vectors in $B' \leq 2^{n-k}$

Therefore $\lvert R' \rvert = \lvert A' \rvert \cdot \lvert B' \rvert \leq 2^n$

Thus choosing $\mu$ = the uniform distribution over $f^{-1}(1)$ 

$\mu(R') \leq \frac{2^n}{2^n + 2^{n - 1} \cdot (2^n - 1)} < \frac{1}{2^{n-1}} \forall R'$ (\text{1-monochromatic rectangles})

Thus $RS(JIP_n) = \frac{1}{2^{n-1}}$

Hence by the rectangle size bound $N(JIP_n) = \lceil log(2^{n-1}) \rceil \geq n$

Thus  $N(JIP_n) = \Omega(n)$

\newpage

\question{5}{Relative Primality}

\problem
Alice and Bob’s inputs are integers a and b, respectively, where $a, b \in [1, 2^n]$. Prove that $\Theta(n/ log n)$ bits are necessary and sufficient to verify nondeterministically that a and b are relatively prime.

\analysis
The protocol providing the upper bound uses the prime number theorem, which in this particular instance, translates to the existence of $n/log(n)$ prime numbers that are possible factors of x and y the inputs of Alice and Bob respectively. The lower bound can be shown by a reduction from the Disjointness problem. 

\definitions
Let $RP_n: {\{0, 1\}}^n \times {\{0, 1\}}^n \to \{0, 1\}$

\claim
$N(RP_n) = \Theta(n/log(n))$ 

\proof 
The protocol that achieves the desired upper bound is the determinstic protocol where Alice send over a binary string of length $n/logn$ (where the $i^{th} \text{bit} = 1 \iff$ the $i^{th}$ smallest prime number in the range is a factor of x). Bob then replies with a single bit i.e. the output which is equal to 0 if one of the factors of x is also a factor of y and equal to 1 otherwise. 

Thus $N(RP_n) = O(n/log(n))$

The lower bound can be shown by reduction from the disjointness function on $n/logn$ bits. 

In particular, $DISJ_{(n/logn)}: {\{0, 1\}}^{(n/logn)} \times {\{0, 1\}}^{(n/logn)} \to \{0, 1\}$ where $DISJ_{(n/logn)} = 1 \iff$ the two strings do not have a 1 in the same position, otherwise $DISJ_{(n/logn)} = 0$

$DISJ_{(n/logn)}$ can be reduced to $RP_n$ (as evidenced by the above protocl) and hence if there exists a non-deterministic protocol with cost less than $n/log(n)$ bits for $RP_n$ then there must exist a protocol with cost less than $n/log(n)$ bits for $DISJ_{(n/logn)}$. But this is a contradiction as we have proven in class that $N(DISJ_{(n/logn)}) = n/logn$.

Thus $N(RP_n) = \Omega(n/log(n))$

Hence, in conclusion $N(RP_n) = \Theta(n/log(n))$

\newpage

\question{6}{Orthogonal Subspaces}

\problem
On input linear subspaces $A, B \subseteq \mathbb{F}_2^n$, prove that $\Theta(n^2)$ bits of nondeterministic communication are necessary and sufficient to check if A and B are orthogonal.

\analysis
Using the understanding that the orthogonal complement of a subspace is the subspace that contains all orthogonal vectors there is an intuitive way to construct a large fooling set. 

\definitions
Let $OS_n: {\{0, 1\}}^{n^2} \times {\{0, 1\}}^{n^2} \to \{0, 1\}$

Let $A$ be any subspace in the set of valid inputs

\claim
$N(OS_n) = \Omega(n^2)$ 

\proof 
A valid fooling set is the set of tuples of the form $(A, A^{\perp}) \forall A$

By definition of orthogonal complement, it is the subspace containing \textbf{all} vectors orthogonal to a susbspace, hence one of the crosspoints for any two points must neccessarily be 0 as atleast one of the crosspoints cannot correspond to a subspace of the orthogonal complement. 

Thus the $f_s(OS_n) = \lvert$ Subspaces of $\mathbb{F}_2^n \rvert$

Number of subspaces of $\mathbb{F}_2^n = 1 + {2^n \choose n - 1} + ... + {2^n \choose 1} + 1$

The largest term in the above expression is by far the second term as the terms representing combinations reduce in magnitude are decreasing as $n << 2^{n-1}$

$\therefore f_s(OS_n) = \Omega{2^n \choose n - 1} = \Omega({2^{n^2}})$

Thus $N(OS_n) = \Omega(log(2^{n^2})) = \Omega(n^2)$

The trivial protocol shows that this bound is tight. 

\newpage

\question{7}{Communication v/s Randomness}

\problem
Prove that any randomized protocol for EQn with probability of correctness 2/3 and communication cost c must use more than $log(n/c)$ bits of randomness.

\definitions
Let $EQ_n: {\{0, 1\}}^{n^2} \times {\{0, 1\}}^{n^2} \to \{0, 1\}$

Let P be a particular randomized protocol for $EQ_n$ with cost c with probability of correctness = 2/3

Let $B_P$ = number of random bits used by protocol P 

\claim
$B_{P} \geq log(n/c)$

\analysis
I was not able to come up with any reasonable proof for the claim, however my approach was to attempt to show that any protocol that does better than this would imply the existence of a \emph{randomized (or so to say)} protocol with error = 0 with constant randomized cost by using the technique to minimize randomness alongside that for error minimization (as discussed in class). 

\newpage
\question{8}{Better Than Random}

\problem
Prove that every $f : {\{0, 1\}}^n \times {\{0, 1\}}^n \to \{0, 1\}$ has a randomized protocol with constant cost and error at most $1/2 - \Theta(2^{-n/2})$

\definitions
Let $EQ_n: {\{0, 1\}}^{n^2} \times {\{0, 1\}}^{n^2} \to \{0, 1\}$

Let P be a particular randomized protocol for $EQ_n$ with cost c with probability of correctness = 2/3

Let $B_P$ = number of random bits used by protocol P 

\claim
$B_{P} \geq log(n/c)$

\analysis
I was not able to come up with any reasonable proof for this problem either, however, my approach was to attempt to show that for any communication problem with a fooling set of size $2^n$ it is possible to construct a randomized protocol of constant cost with the aformentioned error, and thus every communication problem must be amenable to such a randomized protocol. 

\newpage
\newAppendix{Appendix:}{Python Script used for Empirical Data for Rank in Various Fields}
\begin{lstlisting}
from sage.all import *
    num_of_bits = int(raw_input("How many bits?\n"))
    m_f = []
    for x in range(0, 2**num_of_bits):
        l1 = []
        for y in range(0, 2**num_of_bits):
            z = x * y
            binary = ("{0:b}".format(z))[::-1]
            if (len(binary) < num_of_bits):
                l1.append(0)
            else:
                l1.append(int(binary[num_of_bits - 1]))
        m_f.append(l1)
    field_num = 2**(num_of_bits - 1)
    print(2*field_num)
    matrix_f = (matrix(GF(2), m_f))
    print(matrix_f.rank())
    matrix_f = (matrix(GF(2**num_of_bits), m_f))
    print(matrix_f.rank()) 
    matrix_f = (matrix(GF(2**(num_of_bits+1)), m_f))
    print(matrix_f.rank())
    matrix_f = (matrix(m_f))
    print(matrix_f.rank())
    kernel_f = (kernel(matrix_f)).matrix()
    print(kernel_f)
    hypothesis = True
    row_num = 0
    for row in kernel_f:
        pos = 0
        sum = 0
        abs_sum = 0
        digits = 0
        for element in row:
            sum += ((pos*element))
            abs_sum += abs((pos * element))
            pos += 1
            if element != 0:
                digits += 1
        if sum != 0:
            hypothesis = False
            break
        print(str(row_num) + ":" + str(digits) + " " + str(abs_sum))
        row_num += 1
    print(hypothesis) 
       
\end{lstlisting}
\end{document}










