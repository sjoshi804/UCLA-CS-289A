%=======================02-713 LaTeX template, following the 15-210 template==================
%
% You don't need to use LaTeX or this template, but you must turn your homework in as
% a typeset PDF somehow.
%
% How to use:
%    1. Update your information in section "A" below
%    2. Write your answers in section "B" below. Precede answers for all 
%       parts of a question with the command "\question{n}{desc}" where n is
%       the question number and "desc" is a short, one-line description of 
%       the problem. There is no need to restate the problem.
%    3. If a question has multiple parts, precede the answer to part x with the
%       command "\part{x}".
%    4. If a problem asks you to design an algorithm, use the commands
%       \algorithm, \correctness, \runtime to precede your discussion of the 
%       description of the algorithm, its correctness, and its running time, respectively.
%    5. You can include graphics by using the command \includegraphics{FILENAME}
%
\documentclass[11pt]{article}
\usepackage{amsmath,amssymb,amsthm}
\usepackage[noend]{algpseudocode}
\usepackage{graphicx}
\usepackage[margin=1in]{geometry}
\usepackage{mathtools}
\usepackage{fancyhdr}
\usepackage{algorithmicx}
\usepackage{listings}
\setlength{\parindent}{0pt}
\setlength{\parskip}{5pt plus 1pt}
\setlength{\headheight}{13.6pt}
\newcommand\question[2]{\vspace{.25in}\hrule\textbf{#1: #2}\vspace{.5em}\hrule\vspace{.10in}}
\newcommand\newAppendix[2]{\vspace{.25in}\hrule\textbf{#1 #2}\vspace{.5em}\hrule\vspace{.10in}}
\renewcommand\part[1]{\vspace{.10in}\textbf{(#1)}}
\newcommand\analysis{\vspace{.10in}\emph{Analysis: }\newline}
\newcommand\outline{\vspace{.10in}\textbf{Proof Outline: }\newline}
\newcommand\algorithm{\vspace{.10in}\emph{Algorithm: }}
\newcommand\correctness{\vspace{.10in}\emph{Correctness: }\newline}
\newcommand\runtime{\vspace{.10in}\emph{Running time: }\newline}
\newcommand\problem{\emph{Problem Statement:}\newline}
\newcommand\definitions{\emph{Definitions:}\newline}
\newcommand\claim{\emph{Claim.}\newline}
\pagestyle{fancyplain}
\lhead{\textbf{\NAME}}
\chead{\textbf{CS 289A Midterm Problem Set}}
\rhead{\today}
\begin{document}\raggedright
%Section A==============Change the values below to match your information==================
\newcommand\NAME{Siddharth Joshi}  % your name
\newcommand\HWNUM{}              % the homework number
%Section B==============Put your answers to the questions below here=======================

% no need to restate the problem --- the graders know which problem is which,
% but replacing "The First Problem" with a short phrase will help you remember
% which problem this is when you read over your homeworks to study.

\question{5}{Hamming Distance} 

\problem
Let $PRIME_n$ be the problem of determining whether the Hamming distance between two given n-bit strings is a prime. \newline
Prove that $R_{1/3}(PRIME_n) = \Omega$(n).

\analysis
This problem seem nearly insurmountable to tackle from first principles and thus a natural approach is to attempt to reduce it to a known problem. In particular, this solution discusses a possible attempt to prove the bound using a reduction from the Disjointness problem, the only problem covered in class with $\Omega(n)$ randomized complexity while not being entirely intractable to the models covered in class. While I wasn't able to definitively show a provably correct reduction from the disjointness problem to the prime gap hamming distance problem, the solution discusses potential ideas for this. \newline

\definitions
Let $DISJ_n(x, y)$ = 1 if x and y are disjoint and 0 otherwise \newline

\claim
$R_{1/3}(PRIME_n) = \Omega(n)$ \newline

\proof 
Idea for reduction from $DISJ_n$ to $PRIME_n$: \newline
Run the $PRIME_n$ protocol on the strings and if the gap hamming distance is prime, strings are more likely to be disjoint (was not able to prove this, but only able to observe from empirical testing). \newline
If the assumption stands, then the protocol is better than random for prime numbers, and thus a random coin toss for non prime numbers while forcing us to incur significant amount of error (in particular error exponentially close to random guessing) but nonetheless provides a meaningful reduction. \newline

During my research for this problem, I was able to come across a solution that proves the linear bound for the Index Problem where $INDEX_n(x, y)$ = x-th bit of y, and then goes on to reduce the Index Problem to the Gap Hamming Distance problem (which in turn can be used to obtain a simple reduction to the prime version discussed here). However, the reduction used incurs some error as it uses shared randomness to achieve its goals. 
\newpage

\question{9}{The Hat Problem} 

\problem
A group of n prisoners are brought into a room with n chairs arranged
in a circle. Once they are all seated, each prisoner has a hat placed on their head. There
are n distinct colors of hats, which are known to the prisoners beforehand, though colors
may be repeated or not appear on any prisoner’s head. Each prisoner can see everyone’s hat
color except for their own. The prisoners must then simultaneously call out a guess for their
own hat color. The prisoners are set free if at least one of them guesses their own hat color
correctly, and are all executed otherwise. Is it possible for the prisoners to guarantee that
they will all be set free? The prisoners can devise a strategy before entering the room, but
cannot communicate thereafter.

\analysis
The idea is to have the prisoners co-ordinate in a way such that the guesses account for a maximal number of scenarios. In particular, this protocol attempts to have the prisoners guess all possible values that the sum of the hats colors (if colors were interpreted as numbers from 1 to n) could be, and in turn ensures that (exactly) one prisoner guesses correctly, thus guaranteeing success.
\newline

\definitions
Let the colors be interpreted as numbers from 1 through n (any arbitrary ordering suffices)\newline
Let $c_i$ refer to the color of the i-th prisoner's hat \newline
Let $g_i$ refer to the guess made by the i-th prisoner \newline
\newline

\claim
$$ Let g_i = i - \sum_{\forall j \neq i}{c_j} (mod n) $$
$$ \exists i \text{ such that } g_i = c_i $$
\newline

\proof 
The proof is simple, since every prisoner assumes the sum of the colors to be equivalent modulo n to their index and since there are n prisoners, every possible value that the sum of colors could be (modulo n) is considered and thus exactly one must guess the value correctly. Moreover, since there are n colors, exactly one color will be equivalent to the above expression and thus each prisoner is able to make the only valid guess (under their private assumption regarding the value of the sum of the colors of the hats). 

\newpage

\begin{thebibliography}{1}
\bibitem{} Roughgarden, Tim. Lecture 2: Lower Bounds for One-Way Communication: Disjointness, Index, and Gap-Hamming. Stanford, 15 Jan. 2015, theory.stanford.edu/~tim/w15/l/l2.pdf.
\end{thebibliography}

\end{document}










